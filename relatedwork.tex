%!TEX root = paper.tex


\iffalse
In order to improve the performance of the recommender system, users are
encouraged
to go through a preference elicitation process, where they are asked to indicate
their preferences on a sequence of items that they may have consumed before. The
sequence of items in this process is designed such that a minimal effort is
required from a user to get the sufficient number of ratings~\cite{r50,r51,r52}. 
In a closely related work~\cite{r53}, the authors suggested collecting rating on a
group of items as an alternative
to ratings on individual items and used these ratings to assign a new user to a
set of pre-built recommendation profiles. However, a key difference is that in
this work we develop fully personalized recommendation methods and are not
limited to new users as we can include the user's existing  rated items in
the group.

There has been little prior work on using ratings on sets of items in recommender
systems. Most of that work has focused on recommending lists of items or bundles of
items~\cite{xie2010breaking}. Some of these~\cite{r55,moore2012learning,aizenberg2012build} have developed methods to recommend the music
playlists to users. Similarly, the authors
in~\cite{interdonato2013versatile,r54,liu2011personalized,xie2011comprec} presented methods to recommend travel packages to users. However, unlike
music playlists and to some extend travel packages, the items in a list need not
be consumed sequentially, and a user may consume them in no particular
order or consume a few of them, e.g., lists of books, bundle of video games 
and shopping lists. In List Recommendation Model
(LIRE)~\cite{r56}, a user's preference over individual books in a list are used to
recommend existing reading lists to the users.
Unlike these approaches that are focused on list recommendations, we are 
interested in learning the user's preferences over individual items in the list
from his/her preference on the list. 



In addition, there has been a  recent study in which relative
preference information on different groups of items was collected during a new user
signup process and these preferences were then used to assign a user to a set of
pre-built recommendation profiles~\cite{r53}. 
This approach significantly reduced the time required to learn the user's
preferences in order to generate recommendations for the new user.
The principal difference from this 
approach is that in our work we try to model the user behavior that determines
his/her estimated rating on a set and then use that to develop fully
personalized recommendation methods that are not limited to new users.

\fi

There has been little published work on using set-level ratings to improve the
accuracy of item-level recommendations. The one exception is a recent study in which relative
preference information on different groups of items was collected during a new user
signup process and these preferences were then used to assign a user to a set of
pre-built recommendation profiles~\cite{r53}. 
This approach significantly reduced the time required to learn the user's
preferences in order to generate recommendations for the new user.
The principal difference from this 
approach is that in our work we try to model the user behavior that determines
his/her estimated rating on a set and then use that to develop fully
personalized recommendation methods that are not limited to new users.

In addition, there has been some work that  has focused on recommending
lists of items or bundles of items. For example, recommendation of music
playlists~\cite{r55,moore2012learning,aizenberg2012build}, travel
packages~\cite{interdonato2013versatile,r54,liu2011personalized,xie2011comprec}, reading
lists~\cite{r56} and recommendation of lists under user specified budget
constraints~\cite{xie2010breaking,BenouaretRecsys16}.
However, this research is not directly related to the problems explored in this
paper because our focus is on learning the user's ratings on items in lists from
the ratings that the user provided on these lists.

Another relevant work is the problem of energy disaggregation~\cite{hart1992nonintrusive}, 
which refers to the task
of separating the energy signal of a building into the energy signals of
individual appliances that reside in the building. Disaggregated energy
consumptions are used to provide feedback to consumers,  forecast demands,
design energy incentives and detect appliances'
malfunction~\cite{froehlich2011disaggregated,darby2006effectiveness}. Similar to
the idea of energy disaggregation, in our work, we try to separate a user's rating on a set of
items into the users' ratings on items in the set and generate item
recommendations for the user.

The researchers have also investigated how different aspects, e.g., rating
questions~\cite{bellogin2014magic}, reference
points~\cite{adomavicius2011recommender,cosley2003seeing,nguyen2013rating}, and
contextual factors~\cite{Winoto2010RUM}, can influence a user when
elicited to provide a rating on an item. In our work, we have investigated how
does the user provides a rating on a set of items and used the derive insights to develop
collaborative filtering-based methods to predict the rating for an individual
item in the set.

Matrix factorization (MF) is one of the widely used collaborative
filtering-based methods 
in recommender systems~\cite{hu2008collaborative,koren2008factorization,koren2010collaborative,koren2009matrix}. 
The  MF method assume that the
user-item rating matrix is low-rank and can be computed as a product of two
matrices known as the user and the item latent factors. 
If for user $u$, the vector $\bm{p_u} \in \mathbb{R}^f$ denotes the $f$ dimensional
user's latent representation and similarly for item $i$, the vector 
$\bm{q_i} \in \mathbb{R}^f$ represents the $f$ dimensional item's latent
representation, then the
predicted rating for user $u$ on item $i$, i.e., $\hat{r}_{ui}$ is given by
%
\begin{equation} \label{mf_eq}
  \begin{split}
    \hat{r}_{ui} = \bm{p_u}^T\bm{q_i}.
  \end{split}
\end{equation}


The user and item latent factors are learned by minimizing a regularized
square loss between the actual and predicted ratings 
\begin{equation}
  \begin{aligned}
    \operatornamewithlimits{minimize}_{P, Q} & &\frac{1}{2}\sum_{r_{ui} \in R}
    \left(r_{ui} - \bm{p}_u^T \bm{q}_i \right)^2+ \frac{\beta}{2}
    \left(||P||_F^2 + ||Q||_F^2 \right),
  \end{aligned}
\end{equation}
where the matrices $P \in \mathbb{R}^{m \times f}$ and $Q \in\mathbb{R}^{n \times f}$ 
contain the latent factors of the users and the items
respectively. The parameter $\beta$ controls the Frobenius norm regularization
of the latent factors to prevent overfitting. 
Equation~\ref{mf_eq} can be solved by using Stochastic Gradient Descent (SGD)
method~\cite{koren2009matrix}.

\iffalse
Since naive SGD is not easy to parallelize and does not scale for large datasets, a
variety of methods have been proposed to parallelize
SGD~\cite{gemulla2011large,recht2011hogwild,recht2013parallel}.
Alternating Least Squares (ALS)~\cite{zhou2008large} and CCD++~\cite{yu2012scalable} 
can also be used for matrix factorization, and unlike SGD these
schemes are inherently suitable for parallelization.
\fi
