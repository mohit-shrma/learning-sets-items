%!TEX root = paper.tex

\subsection{Dataset}\label{EVALDATA}
We evaluated the proposed methods on two datasets: (i) the dataset analyzed in
Section~\ref{dataset}, which will be referred to as \MLREALSETS, and (ii) a set
of synthetically generated datasets that
allow us to assess how well
the optimization algorithms can estimate accurate models and how their
accuracy depends on various data characteristics.

The synthetic datasets were derived from the \ML
20M
dataset\footnote{https://grouplens.org/datasets/movielens/20m/}~\cite{harper2016movielens}  which
contains 20 million ratings from approximately 229,060 users on 26,779 movies.
For experiment purposes, we
created a synthetic low-rank matrix of rank 5 as follows. 
%
We started by generating two matrices
$A\in\mathbb{R}^{n\times k}$ and $B\in\mathbb{R}^{m\times k}$, where $n$ is
number of users, $m$ is number of items and $k = 5$, whose values are
uniformly distributed at random in $[0, 1]$. We then computed the singular value
decomposition of these matrices to obtain $A=U_A\Sigma_A V_A^T$ and $B=U_B\Sigma_B
V_B^T$. We then let $P=\alpha U_A$, $Q=\alpha U_B$ and $R = PQ^T$. Thus, the final
rank $k$ matrix $R$ is obtained as the product of two randomly generated rank $k$
matrices whose columns are orthogonal. Note that the parameter $\alpha$ was
determined empirically in order to produce ratings in the range of $[-10, 10]$.

Since we know the complete synthetic low-rank matrix we can generate the rating
corresponding to an observed user-item pair in the real dataset from the complete
rating matrix.
We randomly selected 1000 users without replacement from the dataset and for
each user we created sets containing five movies. 
The movies in a user's set
were selected at random without replacement from the movies rated by that user.
For each user, we created at least $k$ such sets of movies, where $k \in [40,
60, 80, 100, 140]$.
We generated rating for a user on a set by 
following two approaches: 
\begin{enumerate}[(i)]  
  \item \ES-based rating: For each user, we chose one of the
extremal subsets at random and used that to generate ratings for all the sets.
The set is assigned an average of the user's ratings on the items in the
chosen extremal subset of the items in the set. 

  \item  \VO-based rating: For each user, 
we chose the user's level of pickiness (the $\beta_u$ parameter) at random from the
range [-2.0, 2.0]. The set is assigned an average of the user's ratings on the items in the set,
and also we offset this rating by adding a quantity computed by scaling the
standard deviation of ratings in the set by the randomly chosen user's level of
pickiness.
\end{enumerate}

For all these datasets, we added random $\mathcal{N}(0, 0.1)$ Gaussian noise while computing
ratings at both the item and set-level for the users. 
For each approach, we generated 15 different synthetic datasets, each
by varying the user-item latent factors and the users' pickiness.

\iffalse
We generated five ratings for
a set as per above approaches by choosing at random five different extremal subsets 
and pickiness levels for the users. We repeated this procedure three times by
using a different low rank model and sampling different 1000 users each time, 
hence creating 3 different synthetic datasets. Hence, giving us fifteen splits
for data generated using each of the above approach.  We will refer to these datasets
as \SYNSETSA.
\fi
%Various statistics about the set and item ratings are shown in Table~\ref{table:dataset_stats}.


%\subsection{Comparison methods}
%We compared the proposed method against the \MF (MF)~\cite{r43} and the
%Bayesian Personalized Ranking (BPR)~\cite{r6} methods using the ratings available
%for the items in the sets.   
%\footnote{Code for the developed methods and baselines is available at https://sites.google.com/site/lfsicdm16/}


\subsection{Evaluation methodology}
To evaluate the performance of the proposed methods we divided the available
set-level ratings for each user into training, validation and test splits by randomly
selecting five set-level ratings for each of the validation and test splits. 
The validation split was used for model selection.
In order to assess the performance of the methods for item
recommendations, we used a test set that contained for each user the items that
were not present in the user's sets (i.e., these were absent from the training,
test, and validation splits) but were present in the original user-item rating
matrix used to generate the sets.
We used Root Mean Square Error (RMSE) to measure the accuracy of the rating prediction
over items and sets. 

\subsubsection{Comparison methods}\label{comp_methods}
In addition to the evaluation of the proposed methods, i.e., ARM, \ES 
and \VO, we also present the results for the following non-personalized
methods:
\begin{enumerate}[(i)]
  \item SetAvg: This method predicts a user's ratings on items and sets as the average of the
    user's ratings on sets. The rating of user $u$ on set $\mathcal{S}$ is given
    by
    \begin{equation}
      \begin{split}
        \hat{r}_{u}^s &= \frac{1}{|\mathcal{Q}_u|} \sum_{k \in \mathcal{Q}_u} \hat{r}_{uk}, 
      \end{split}
    \end{equation}
  \noindent where $\mathcal{Q}_u$ represents all the sets rated by user $u$. The
  rating of user $u$ on item $i$ is given by
  \begin{equation}
    \begin{split}
      \hat{r}_{ui} &= \frac{1}{|\mathcal{Q}_u|} \sum_{k \in \mathcal{Q}_u}
      \hat{r}_{uk},
    \end{split}
  \end{equation}
\noindent where $\mathcal{Q}_u$ represents all the sets rated by user $u$.

  \item Item average: This method estimates the rating for an item as the average of the ratings
provided by the users on the item. The rating
$\hat{r_i}$ for an item $i$ is given by
\begin{equation}
  \begin{split}
    \hat{r}_{i} &= \frac{1}{|\mathcal{U}_i|}\sum_{u \in \mathcal{U}_i} r_{ui}, 
  \end{split}
\end{equation}
\noindent where $\mathcal{U}_i$ denotes the set of users who have rated item
$i$. 
 

  \item UserMeanSub: This method estimates the rating for an item as the sum of average rating on
sets and average of user mean subtracted item ratings.
The rating
$\hat{r_i}$ for an item $i$ is given by
\begin{equation}
  \begin{split}
    \hat{r}_{i} &= \mu_s + \frac{1}{|\mathcal{U}_i|}\sum_{u \in \mathcal{U}_i}
    \big(r_{ui} - \frac{1}{|\mathcal{I}_u|}\sum_{k \in \mathcal{I}_u}r_{uk}\big),
  \end{split}
\end{equation} 
\noindent where $\mu_s$ is the average of the ratings on all the sets, $\mathcal{I}_u$
represents the set of items rated by user $u$.

\end{enumerate}

In practice, 
a significant proportion of the ratings provided by users on items
depends on factors that are associated with either users or items, and do not
depend on interactions between the users and the items. For example, some users
have a tendency to rate higher than others, and some items receive higher
ratings than others. For the real set-level rating dataset, that we obtained from
\ML users, we determined these factors by estimating user- and
item-biases~\cite{koren2009matrix} as part of the model learning.


%\begin{table}[t]
%  \centering
%  \caption{Statistics for ratings available for items in and outside sets.} \label{table:dataset_stats}
%  \begin{tabular}{|l|c|c|c|}
%    \hline
%      &\multicolumn{1}{|p{1cm}|}{\centering ratings}
%      &\multicolumn{1}{|p{1.5cm}|}{\centering ratings/user}
%      &\multicolumn{1}{|p{1.5cm}|}{\centering ratings/item} \\ 
%    \hline
%    Items in sets & 146,894 & 172.0 & 11.7 \\
%    Items outside sets & 310,664 & 363.78 & 24 .76 \\
%  \hline
%  \end{tabular}
%\end{table}


\iffalse
We followed the above procedure on all the splits of synthetic datasets
generated using the approaches described in Section~\ref{EVALDATA} and report average value
of the performance assessment metrics for each of the approaches. 
Similarly we repeated the above process five times for \MLREALSETS  and report
the average value of metrics across the runs.  
\fi



\subsection{Model selection}
We performed grid search to tune the dimensions of the latent factors and
regularization hyper-parameters for the latent factors. 
We searched
for regularization weights ($\lambda$) in the range [0.001, 0.01, 0.1, 1, 10],
$\epsilon$ in the range [0.1, 0.25, 0.5, 1] and $c$ in the range [0, 0.25, 0.50,
0.75, 0.90] for both the synthetic and the real
datasets.
We searched for the dimension of latent factors ($f$) in the range [1, 5, 10,
15, 25, 50, 75, 100] for real datasets, and used 5 as the dimension
of latent factors for synthetic datasets.
The final
parameters were selected based on the performance on the validation split. 

